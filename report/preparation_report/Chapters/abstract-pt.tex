%!TEX root = ../template.tex
%%%%%%%%%%%%%%%%%%%%%%%%%%%%%%%%%%%%%%%%%%%%%%%%%%%%%%%%%%%%%%%%%%%%
%% abstrac-pt.tex
%% NOVA thesis document file
%%
%% Abstract in Portuguese
%%%%%%%%%%%%%%%%%%%%%%%%%%%%%%%%%%%%%%%%%%%%%%%%%%%%%%%%%%%%%%%%%%%%

\typeout{NT FILE abstrac-pt.tex}%

De forma a abordar os \textit{tradeoffs} relacionados com escala, segurança e desempenho, assim como a capacidade de resposta do sistema na finalização de blocos, foram propostas soluções baseadas em modelos de consenso híbrido para registos distribuídos com grande capacidade de escalabilidade. A ideia principal do Consenso Híbrido é construir um consenso eficiente, escalável e seguro em que é composto por uma combinação de dois tipos de consensos, um mecanismo lento e ineficiente (\textit{PoW} ou outros protocolos de consenso inspirados em Nakamoto) com um consenso eficiente normalmente aplicado em sistemas privados, o qual é executado com base em comités formados dinamicamente com características resistentes a \textit{sybil attacks}. Deste modo, planos de consenso híbrido podem utilizar \textit{PoW} como base para eleger comités. Por sua vez, os comités são utilizados para executar modelos de consenso com o objetivo principal de ordenar transações.

Nas abordagens atuais de Consenso Híbrido, o plano de consenso é direcionado para implementar uma forma específica de eleição de um comité para um modelo de consenso específico. Em geral, este é otimizado para aplicações específicas de criptomoedas, o qual não permite outro tipo de aplicações diversificadas que expressem outras formas de consenso, utilizadas como alternativas ou usadas em combinação.

Nesta dissertação propomos o desenho, implementação e avaliação experimental da plataforma \mysystem: uma arquitetura baseada em registos públicos que suporta um plano de consenso híbrido e flexível para participantes anónimos, o qual executa com resistência a \textit{sybil attackes}. Para a nossa solução esperada, a ideia é construir um plano de serviço de consenso híbrido modular baseado numa arquitetura base de \textit{blockchain}. \mysystem~irá expor \textit{APIs} para suportar transações com \textit{on-chain smart-contracts}, os quais necessitarão de expressividade necessária para a possibilidade de alternar entre os modelos de consenso ou um uso combinado dos mesmos que serão disponibilizados pelo plano de serviço de consenso projetado.

\keywords{Registos Descentralizados \and Registos consistentes e replicados \and \textit{Blockchains} públicas \and \textit{Blockchains} paralelas \and Modelos de consenso \and Plano de consenso Híbrido e Flexível \and Resistência a ataques de \textit{Sybil} \and Consenso Flexível com participantes anónimos \and \textit{Smart Contracts} \and Segurança \and Confiável }
% to add an extra black line
